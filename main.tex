\documentclass[12pt]{article}
\usepackage{enumitem}
\usepackage[margin=0.5in]{geometry}
\usepackage{amsfonts}
\usepackage[normalem]{ulem} % [normalem] prevents the package from changing the default behavior of `\emph` to dashuline.


\renewcommand{\familydefault}{\sfdefault}

\usepackage{xcolor}
\pagecolor[rgb]{0.15625,0.15625,0.15625} %black

\color[rgb]{0.69921875,0.69921875,0.69921875} %grey

\begin{document}

\section*{Precalc notes}

\begin{enumerate}
    \item \begin{enumerate}
              \item $x$
              \item $-4$
              \item $\frac 3 2$
              \item $\sqrt 5$
          \end{enumerate}
    \item \begin{enumerate}
              \item $ab=ba$; associative property
              \item $a + (b + c) = (a + b) + c$; commutative property
              \item $a(b + c) = ab + ac$; Distributive property
          \end{enumerate}
    \item \begin{enumerate}
              \item $A = \{x \mid x \in \mathbb{R} \mid 2 < x < 7 \}$
              \item $(2,7)$
          \end{enumerate}
    \item
          \begin{enumerate}
              \item The symbol $x$ stands for the \dashuline{absolute value} of the number x.
                    \newline If $x$ is not 0, then she sign of $|x|$ is always \dashuline{positive}
          \end{enumerate}
    \item The distance between $a$ and $b$ on the real line is $d(a,b) = \dashuline{|a-b|}$.
          \newline So the distance between $-5$ and $2$ is \dashuline{7}.
    \item \begin{enumerate}
              \item yes
              \item yes
          \end{enumerate}
    \item \begin{enumerate}
              \item ...
              \item ...
          \end{enumerate}

\end{enumerate}
\section*{1.4 }

\end{document}